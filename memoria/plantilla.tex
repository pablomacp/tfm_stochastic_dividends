\documentclass[11pt,a4paper,spanish]{book}
\usepackage{estilo_unir}



%---------------------------
%título del trabajo y autor
%---------------------------
\title{Análisis de un modelo de dividendos estocásticos en la valoración de derivados financieros}
\author{Pablo Macías Pineda}
\date{15 de junio de 2020}
\director{María Luisa Díez Platas}
\nombreciudad{Madrid}

%---------------------------
%marges
%---------------------------
%\usepackage[margin=1.9cm]{geometry}
%---------------------------
%---------------------------
%---------------------------
%---------------------------
\begin{document}
\renewcommand{\listfigurename}{Índice de Ilustraciones}
\renewcommand{\listtablename}{Índice de Tablas}
\renewcommand{\contentsname}{Índice de Contenidos}
\renewcommand{\figurename}{Figura}
\renewcommand{\tablename}{Tabla} 

\maketitle

\frontmatter
\tableofcontents
\listoffigures
\listoftables

\chapter{Resumen}
{\bf Nota:} En este apartado se introducirá un breve resumen en español del trabajo realizado (extensión máxima: 150 palabras). Este resumen debe incluir el objetivo o propósito de la investigación, la metodología, los resultados y las conclusiones.


{\bf Palabras Clave:} Se deben incluir de 3 a 5 palabras claves en español

\chapter{Abstract}
{\bf Nota:} En este apartado se introducirá un breve resumen en español del trabajo realizado (extensión máxima: 150 palabras). Este resumen debe incluir el objetivo o propósito de la investigación, la metodología, los resultados y las conclusiones.


{\bf Palabras Clave:} Se deben incluir de 3 a 5 palabras claves en inglés


\mainmatter

\chapter{Introducción}
\input{1.0-Introduccion}
\cleardoublepage

\chapter{Contexto y estado del arte}
\input{2.0-Contexto}
\cleardoublepage

\chapter{Identificación de requisitos}
\input{3.0-Requisitos}
\cleardoublepage

\chapter{Objetivos}
\input{4.0-Objetivos}
\cleardoublepage

\chapter{Desarrollo del trabajo}
\input{5.0-Desarrollo}
\cleardoublepage

\chapter{Conclusiones y trabajo futuro}
\input{6.0-Conclusiones}
\cleardoublepage


\begin{thebibliography}{a}

\bibitem{Shreve} \textsc{Steven E. Shreve},
\textit{Stochastic Calculus for Finance II. Continuous-Time Models.}
Información addicional

\bibitem{Hull} \textsc{John C. Hull},
\textit{Options, futures and other derivatives.}
Información addicional

\bibitem{Bergomi} \textsc{Lorenzo Bergomi},
\textit{Stochastic Volatility Modeling.}

\end{thebibliography}
%\bibliographystyle{plain} 
%\bibliography{bibliografia}

\appendix
\chapter{Apéndices}

\end{document}